\documentclass{standalone}%
% Draw the setup where the source and detector move, e.g. classic CT
% With help from https://tex.stackexchange.com/q/515519/828
\usepackage{fontawesome5}
\usepackage{ifthen}
\ifthenelse{\isundefined{\everyframe}}{%
	% If we're compiling this file via \input, then we already defined some things
	% In the other case, we need to define them
	\usepackage{tikz}
	\usepackage{animate}
	\newcommand{\everyframe}{5}
	\definecolor{ubRed}{HTML}{e4003c}%
	\definecolor{ubGrey}{HTML}{646363}%
	% split complementary images from https://www.sessions.edu/color-calculator/
	\definecolor{ubRedComplementary}{HTML}{2EE600}
	}{}
\begin{document}
\begin{animateinline}[loop,every=\everyframe]{25}
	\multiframe{90}{n=1+4}{%
		\begin{tikzpicture}[scale=1.25]
			\pgfdeclarelayer{background}
			\pgfsetlayers{background,main}
			%Help lines used to setup the animation (set to semitransparent), drawing them transparent in the presentation forces a consistent size
			\begin{pgfonlayer}{background}
				\draw[ubGrey,transparent,help lines,step=5mm] (-2.05,-2.05) grid (2.05,2.05);
			\end{pgfonlayer}
			% Stuff that stays put
			\node[ubRedComplementary] at (0,0) (sample) {\Huge\faUser};
			% Stuff that moves
				\begin{scope}[rotate around={\n:(sample)}]
				% Rotation arc
				\draw[->, thick,line cap=rect] (1.5,0) arc [start angle=0, end angle=180, radius=1.5];
				\draw[->, thick,line cap=rect] (-1.5,0) arc [start angle=-180, end angle=0, radius=1.5];
				% Source
				\fill[ubRed] (-0.25,1.5) rectangle node (source) {} +(0.5,0.5);
				\draw[fill=yellow] (0,1.75) circle (0.2);
				\node at (0,1.735) (radiation) {\small\faRadiation};
				% Detector and detector edges
				\fill[gray] (-0.5,-1.75) rectangle node (detector) {} +(1,0.25);
				\coordinate (dl) at (-0.45,-1.75);
				\coordinate (dr) at (0.45,-1.75);
				% X-ray cone
				\begin{pgfonlayer}{background}
					\fill[gray,semitransparent] (source.center) -- (dl) -- (dr) -- cycle;
				\end{pgfonlayer}
				\end{scope}
		\end{tikzpicture}
	}
\end{animateinline}
\end{document}
